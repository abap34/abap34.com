z

% 基本パッケージ
\usepackage{multicol}
\usepackage[top=15mm,bottom=15mm,left=18mm,right=18mm]{geometry} % 余白縮小
\usepackage{titlesec}
\usepackage{hyperref}
\usepackage{enumitem}
\usepackage{otf}
\usepackage{parskip}
\usepackage{xcolor}
\usepackage{microtype} % タイポグラフィの微調整
\usepackage{graphicx} % スケーリング用


% FontAwesome パッケージ読み込み
\usepackage{fontawesome}

% XeLaTeX / LuaLaTeX 前提で
\usepackage{fontspec}
% FontAwesome のシンボルを出力するためのフォントファミリ定義
\newfontfamily{\FA}{FontAwesome}[ % フォント名は環境によって "FontAwesome Regular" などになることも
  UprightFont = {* Regular},         % フォントファイル名が "FontAwesome Regular.otf" の場合
  Extension = .otf                   % or .ttf
]

\setlength{\columnsep}{2.5em}
\setlength{\columnseprule}{0.05pt} % 真ん中の線の太さ
\definecolor{rulecolor}{RGB}{220, 220, 220} 
\renewcommand{\columnseprulecolor}{\color{rulecolor}}
\setlength{\columnwidth}{0.5\textwidth}


\definecolor{accent}{RGB}{0, 59, 92}       % 濃紺
\definecolor{subtext}{RGB}{120, 120, 120}  % やや薄めのグレー
\definecolor{rule}{RGB}{200, 200, 200}     % より薄いグレー

\hypersetup{
    colorlinks=true,
    linkcolor=accent,
    urlcolor=accent,
    pdfborderstyle={/S/U/W 0.5}
}


% 1) セクション見出しのフォントサイズを \Large に
\titleformat{\section}
  {\normalfont\Large\bfseries\scshape\color{accent}} 
  {}{0pt}{}% 見出し番号やインデントは使わない
  [
    \vspace{-1.0em}% 見出し直後のルールと文字の間隔を縮める
    \color{rule}\rule{\linewidth}{0.1pt}%
  ]

% 2) 見出しまわりの上下スペースを調整
\titlespacing*{\section}
  {0pt}                            % 左マージン
  {1.5ex plus .2ex minus .2ex}     % 見出し前のスペース
  {0.5ex plus .1ex}                % 見出し後のスペース

% リストスタイル
\setlist[itemize]{leftmargin=1.5em,label=\small\textendash,itemsep=0.3em,parsep=0.1em}

% パラグラフ間の間隔調整
\setlength{\parskip}{0.3em}

\pagestyle{empty}

\begin{document}

% ヘッダー(左寄せでアカデミックなデザイン)
{\Huge\bfseries Yuchi Yamaguchi}

% ——— プロフィールヘッダー ———
\noindent
\begin{minipage}[t]{0.48\linewidth}
  \raggedright
  {\FA\faEnvelope\ }\texttt{uchi.ymgc@gmail.com} \\[0.3em]
  {\FA\faGithub\ }\href{https://github.com/abap34}{abap34}
\end{minipage}%
\hfill
\begin{minipage}[t]{0.48\linewidth}
  \raggedright
  {\FA\faLink\ }\href{https://abap34.com}{abap34.com} \\[0.3em]
  {\FA\faMapMarker\ }Kanagawa, Japan
\end{minipage}



{\color{rule}\rule{\linewidth}{0.15pt}}



% ここから2カラム開始
\begin{multicols}{2}

% 左カラム

% 教育
\section*{Education}
\textbf{Institute of Science Tokyo} — Bachelor of Engineering \\
{\small School of Computing, Department of Computer Science} \\
{\footnotesize\textit{Apr 2022 – Mar 2026 (expected)}}


% 職歴
\section*{Experience}

\textbf{DENSO IT Laboratory} — Research Intern \\
{\footnotesize\textit{Jul 2022 – Sep 2022}} \\
{\small - Conducted research on anomaly detection using deep learning techniques}


\textbf{XICA Inc.} — Software Engineer Intern \\
{\footnotesize\textit{Feb 2023 – Apr 2023}} \\
{\small - Improved performance of an existing Python-based data analysis infrastructure}


\textbf{Nikkei Inc.} — Research Intern \\
{\footnotesize\textit{Nov 2024 – Present}} \\
{\small 
- Designed problems for data analysis competitions \\
- Conducted research on the reproducibility of machine learning papers
}

% スキル
\section*{Skills}
{\small
\textbf{Programming:} Python, Julia, C++, TypeScript, Scheme, Go, C, Rocq \\
\textbf{Tools:} Git, Docker, PyTorch, Linux \\
\textbf{Languages:} Japanese (native), English (conversational)
}

% 右カラムに行くための列区切り
\columnbreak

% 右カラム

% プロジェクト
\section*{Projects}
\textbf{Dotfiles Setup System} \\
{\small
- Developed a cross-platform environment setup tool using shell scripts \\
- Automatically configures development tools and shell environments
}


% 賞歴(短くシンプルに)
\section*{Awards}
{\small\textit{Your academic honors and awards here}}


% 研究興味
\section*{Research Interests}
{\small
Programming language design, theorem proving, formal methods, \\
machine learning, computational complexity, distributed systems
}


% 出版物(オプション、シンプルに)
\section*{Publications}
{\small\textit{Your publications and conference presentations here}}

\end{multicols}

% フッター(現在の日付に更新)
\vfill
{\footnotesize\color{subtext}\centering Last updated: May 10, 2025}

\end{document}